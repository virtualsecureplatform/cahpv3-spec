\documentclass[a4paper,10pt]{article}

% Package includes

\usepackage{graphicx}
\usepackage{geometry}
\usepackage{array}
\usepackage{colortbl}
\usepackage[svgnames]{xcolor}

\usepackage[colorlinks,citecolor=Navy,linkcolor=Navy]{hyperref}
\usepackage{placeins}
\usepackage{longtable}
\usepackage{multirow}
\usepackage{float}
\usepackage{listings}
\usepackage{comment}
\usepackage{enumitem}
\usepackage{verbatimbox}
\usepackage{amsmath}
\usepackage{amssymb}

\usepackage{tabularx}
\renewcommand\tabularxcolumn[1]{m{#1}}

\usepackage[olditem,oldenum]{paralist}

% Setup margins

\setlength{\topmargin}{-0.5in}
\setlength{\textheight}{9in}
\setlength{\oddsidemargin}{0in}
\setlength{\evensidemargin}{0in}
\setlength{\textwidth}{6.5in}

% Useful macros

\newcommand{\note}[1]{{\bf [ NOTE: #1 ]}}
\newcommand{\fixme}[1]{{\bf [ FIXME: #1 ]}}
\newcommand{\todo}[1]{\marginpar{\footnotesize #1}}

\newcommand{\wunits}[2]{\mbox{#1\,#2}}
\newcommand{\um}{\mbox{$\mu$m}}
\newcommand{\xum}[1]{\wunits{#1}{\um}}
\newcommand{\by}[2]{\mbox{#1$\times$#2}}
\newcommand{\byby}[3]{\mbox{#1$\times$#2$\times$#3}}

\newlength\savedwidth
\newcommand\whline[1]{%
  \noalign{%
    \global\savedwidth\arrayrulewidth\global\arrayrulewidth 1.5pt%
  }%
  \cline{#1}%
  \noalign{\vskip\arrayrulewidth}%
  \noalign{\global\arrayrulewidth\savedwidth}%
}

% Custom list environments

\newlist{tightlist}{itemize}{1}
\setlist[tightlist]{label=\textbullet,nosep}

\newenvironment{titledtightlist}[1]
{\noindent
 ~~\textbf{#1}
 \begin{tightlist}}
{\end{tightlist}}

\newenvironment{commentary}
{	\vspace{-1.5mm}
	\list{}{
		\topsep		0mm
		\partopsep	0mm
		\listparindent	1.5em
		\itemindent	\listparindent
		\rightmargin	\leftmargin
		\parsep		0mm
	}
	\item
	\small\em
	\noindent\nopagebreak\rule{\linewidth}{1pt}\par
	\noindent\ignorespaces
}
{\endlist}

%\newenvironment{discussion}
%{	\vspace{-1.5mm}
%	\list{}{
%		\topsep		0mm
%		\partopsep	0mm
%		\listparindent	1.5em
%		\itemindent	\listparindent
%		\rightmargin	\leftmargin
%		\parsep		0mm
%	}
%	\item
%	\small\em
%	\noindent\nopagebreak\rule{\linewidth}{1pt}\par
%	\noindent\textbf{Discussion:}
%}
%{\endlist}

% Other commands and parameters

\pagestyle{myheadings}
\setlength{\parindent}{0in}
\setlength{\parskip}{10pt}
\sloppy
\raggedbottom
\clubpenalty=10000
\widowpenalty=10000

% Commands for register format figures.

% New column types to use in tabular environment for instruction formats.
% Allocate 0.18in per bit.
\newcolumntype{I}{>{\centering\arraybackslash}p{0.18in}}
% Two-bit centered column.
\newcolumntype{W}{>{\centering\arraybackslash}p{0.36in}}
% Three-bit centered column.
\newcolumntype{F}{>{\centering\arraybackslash}p{0.54in}}
% Four-bit centered column.
\newcolumntype{Y}{>{\centering\arraybackslash}p{0.72in}}
% Five-bit centered column.
\newcolumntype{R}{>{\centering\arraybackslash}p{0.9in}}
% Six-bit centered column.
\newcolumntype{S}{>{\centering\arraybackslash}p{1.08in}}
% Seven-bit centered column.
\newcolumntype{O}{>{\centering\arraybackslash}p{1.26in}}
% Eight-bit centered column.
\newcolumntype{E}{>{\centering\arraybackslash}p{1.44in}}
% Ten-bit centered column.
\newcolumntype{T}{>{\centering\arraybackslash}p{1.8in}}
% Twelve-bit centered column.
\newcolumntype{M}{>{\centering\arraybackslash}p{2.2in}}
% Sixteen-bit centered column.
\newcolumntype{K}{>{\centering\arraybackslash}p{2.88in}}
% Twenty-bit centered column.
\newcolumntype{U}{>{\centering\arraybackslash}p{3.6in}}
% Twenty-bit centered column.
\newcolumntype{L}{>{\centering\arraybackslash}p{3.6in}}
% Twenty-five-bit centered column.
\newcolumntype{J}{>{\centering\arraybackslash}p{4.5in}}

\newcommand{\instbit}[1]{\mbox{\scriptsize #1}}
\newcommand{\instbitrange}[2]{~\instbit{#1} \hfill \instbit{#2}~}
\newcommand{\reglabel}[1]{\hfill {\tt #1}\hfill\ }

\newcommand{\wiri}{\textbf{WIRI}}
\newcommand{\wpri}{\textbf{WPRI}}
\newcommand{\wlrl}{\textbf{WLRL}}
\newcommand{\warl}{\textbf{WARL}}

\newcommand{\unspecified}{\textsc{unspecified}}


\title{CAHPv3 Instruction Set Specification}
\author{Naoki Matsumoto, Ryotaro Banno, Kotaro Matsuoka}
\date{2020/09/21}

\begin{document}

\maketitle
\clearpage

\section{Programmers’Model for CAHPv3 ISA}
\subsection{Integer Registers}
\begin{figure}[H]
    {\footnotesize
    \begin{center}
    \begin{tabular}{p{2in}}
    \instbitrange{15}{0} \\ \cline{1-1}
    \multicolumn{1}{|c|}{\reglabel{x0 / ra(return address)}} \\ \cline{1-1}
    \multicolumn{1}{|c|}{\reglabel{x1 / sp(stack pointer)}}  \\ \cline{1-1}
    \multicolumn{1}{|c|}{\reglabel{x2 / fp(frame pointer)}}  \\ \cline{1-1}
    \multicolumn{1}{|c|}{\reglabel{x3 / s0}} \\ \cline{1-1}
    \multicolumn{1}{|c|}{\reglabel{x4 / s1}} \\ \cline{1-1}
    \multicolumn{1}{|c|}{\reglabel{x5 / s2}} \\ \cline{1-1}
    \multicolumn{1}{|c|}{\reglabel{x6 / s3}} \\ \cline{1-1}
    \multicolumn{1}{|c|}{\reglabel{x7 / s4}} \\ \cline{1-1}
    \multicolumn{1}{|c|}{\reglabel{x8 / a0}} \\ \cline{1-1}
    \multicolumn{1}{|c|}{\reglabel{x9 / a1}} \\ \cline{1-1}
    \multicolumn{1}{|c|}{\reglabel{x10 / a2}} \\ \cline{1-1}
    \multicolumn{1}{|c|}{\reglabel{x11 / a3}} \\ \cline{1-1}
    \multicolumn{1}{|c|}{\reglabel{x12 / a4}} \\ \cline{1-1}
    \multicolumn{1}{|c|}{\reglabel{x13 / a5}} \\ \cline{1-1}
    \multicolumn{1}{|c|}{\reglabel{x14 / t0}} \\ \cline{1-1}
    \multicolumn{1}{|c|}{\reglabel{x15 / t1}} \\ \cline{1-1}
    \multicolumn{1}{c}{16} \\

    \instbitrange{15}{0} \\ \cline{1-1}
    \multicolumn{1}{|c|}{\reglabel{pc}} \\ \cline{1-1}
    \multicolumn{1}{c}{16} \\
    \end{tabular}
    \end{center}
    }
    \caption{CAHPv3 integer register state.}
    \label{gprs}
\end{figure}

\section{24-bit Length Instructions}
\subsection{Integer Computational Instructions}
\subsubsection*{Integer Register-Register Instructions}
\vspace{-0.2in}
\begin{center}
\begin{tabular}{S@{}R@{}R@{}S@{}R@{}O}
\\
\instbitrange{23}{20} &
\instbitrange{19}{16} &
\instbitrange{15}{12} &
\instbitrange{11}{8} &
\instbitrange{7}{0} \\
\hline
\multicolumn{1}{|c|}{0000} &
\multicolumn{1}{c|}{rs2} &
\multicolumn{1}{c|}{rs1} &
\multicolumn{1}{c|}{rd} &
\multicolumn{1}{c|}{opcode} \\
\hline
4 & 4 & 4 & 4 & 8 \\
& src2  & src1 & dest & ADD \\
& src2  & src1 & dest & SUB \\
& src2  & src1 & dest & AND \\
& src2  & src1 & dest & XOR \\
& src2  & src1 & dest & OR  \\
& src2  & src1 & dest & LSL \\
& src2  & src1 & dest & LSR \\
& src2  & src1 & dest & ASR \\
\end{tabular}
\end{center}

\begin{center}
    \begin{tabularx}{\textwidth}{|c|c|c|X|} \hline
      Instruction & Opcode & Formula & \multicolumn{1}{c|}{Description} \\ \hline \hline
      ADD rd, rs1, rs2 & 00000001 & $\mathrm{rd} \leftarrow \mathrm{rs1} + \mathrm{rs2}$ &
      Add rs1 value to rs2 value, and store to rd. \\ \hline
      SUB rd, rs1, rs2 & 00001001 & $\mathrm{rd} \leftarrow \mathrm{rs1} - \mathrm{rs2}$ &
      Subtract rs2 value from rs1 value, and store to rd.  \\ \hline
      AND rd, rs1, rs2 & 00010001 & $\mathrm{rd} \leftarrow \mathrm{rs1} \land \mathrm{rs2}$ &
      Do bit and between rs1 value and rs2 value, and store to rd.  \\ \hline
      XOR rd, rs1, rs2 & 00011001 & $\mathrm{rd} \leftarrow \mathrm{rs1} \oplus \mathrm{rs2}$ &
      Do bit xor between rs1 value and rs2 value, and store to rd.  \\ \hline
      OR  rd, rs1, rs2 & 00100001 & $\mathrm{rd} \leftarrow \mathrm{rs1} \lor \mathrm{rs2}$ &
      Do bit or between rs1 value and rs2 value, and store to rd.  \\ \hline
      LSL rd, rs1, rs2 & 00101001 & $\mathrm{rd} \leftarrow \mathrm{rs1} \ll \mathrm{rs2}$ &
      Do left logical shift rs1 value with width rs2 value, and store to rd. \\ \hline
      LSR rd, rs1, rs2 & 00110001 & $\mathrm{rd} \leftarrow \mathrm{rs1} \gg \mathrm{rs2}$ &
      Do right logical shift rs1 value with width rs2 value, and store to rd. \\ \hline
      ASR rd, rs1, rs2 & 00111001 & $\mathrm{rd} \leftarrow \mathrm{rs1} \ggg \mathrm{rs2}$ &
      Do right arithmetic shift rs1 value with width rs2 value, and store to rd.\\ \hline
    \end{tabularx}
\end{center}

\subsubsection*{Integer Register-Immediate Instructions}
\vspace{-0.2in}
\begin{center}
\begin{tabular}{S@{}R@{}R@{}S@{}R@{}O}
\\
\instbitrange{23}{16} &
\instbitrange{15}{12} &
\instbitrange{11}{8} &
\instbitrange{7}{6} &
\instbitrange{5}{0} \\
\hline
\multicolumn{1}{|c|}{simm10[7:0]} &
\multicolumn{1}{c|}{rs1} &
\multicolumn{1}{c|}{rd} &
\multicolumn{1}{c|}{simm10[9:8]} &
\multicolumn{1}{c|}{opcode} \\
\hline
8 & 4 & 4 & 2 & 6 \\
simm10[7:0] & src  & dest & simm10[9:8] & ADDI \\
simm10[7:0] & src  & dest & simm10[9:8] & ANDI \\
simm10[7:0] & src  & dest & simm10[9:8] & XORI \\
simm10[7:0] & src  & dest & simm10[9:8] & ORI \\
simm10[7:0] & 0000  & dest & simm10[9:8] & LI \\
\end{tabular}
\end{center}

\begin{center}
    \begin{tabularx}{\textwidth}{|c|c|c|X|} \hline
      Instruction & Opcode & Formula & \multicolumn{1}{c|}{Description} \\ \hline \hline
      ADDI rd, rs1, simm10 & 000011 & $\mathrm{rd} \leftarrow \mathrm{rs1} + \mathrm{simm10}$ &
      Add rs1 value to simm10, and store to rd.  \\ \hline
      ANDI rd, rs1, simm10 & 010011 & $\mathrm{rd} \leftarrow \mathrm{rs1} \land \mathrm{simm10}$ &
      Do bit and between rs1 value and simm10, and store to rd. \\ \hline
      XORI rd, rs1, simm10 & 011011 & $\mathrm{rd} \leftarrow \mathrm{rs1} \oplus \mathrm{simm10}$ &
      Do bit xor between rs1 value and simm10, and store to rd.  \\ \hline
      ORI  rd, rs1, simm10 & 100011 & $\mathrm{rd} \leftarrow \mathrm{rs1} \lor \mathrm{simm10}$ &
      Do bit or between rs1 value and simm10, and store to rd.  \\ \hline
      LI rd, simm10   & 110101 & $\mathrm{rd} \leftarrow \mathrm{simm10}$ &
      Load simm10, and store to rd. \\ \hline
    \end{tabularx}
\end{center}

\vspace{-0.2in}
\begin{center}
\begin{tabular}{S@{}R@{}R@{}S@{}R@{}O}
\\
\instbitrange{23}{20} &
\instbitrange{19}{16} &
\instbitrange{15}{12} &
\instbitrange{11}{8} &
\instbitrange{7}{0} \\
\hline
\multicolumn{1}{|c|}{0000} &
\multicolumn{1}{c|}{uimm4[3:0]} &
\multicolumn{1}{c|}{rs1} &
\multicolumn{1}{c|}{rd} &
\multicolumn{1}{c|}{opcode} \\
\hline
4 & 4 & 4 & 4 & 8 \\
& uimm4[3:0]  & src & dest & LSLI \\
& uimm4[3:0]  & src & dest & LSRI \\
& uimm4[3:0]  & src & dest & ASRI \\
\end{tabular}
\end{center}

\begin{center}
    \begin{tabularx}{\textwidth}{|c|c|c|X|} \hline
      Instruction & Opcode & Formula & \multicolumn{1}{c|}{Description} \\ \hline \hline
      LSLI rd, rs1, uimm4 & 00101011 & $\mathrm{rd} \leftarrow \mathrm{rs1} \ll \mathrm{uimm4}$ &
      Do left logical shift rs1 value with width uimm4, and store to rd.  \\ \hline
      LSRI rd, rs1, uimm4 & 00110011 & $\mathrm{rd} \leftarrow \mathrm{rs1} \gg \mathrm{uimm4}$ &
      Do right logical shift rs1 value with width uimm4, and store to rd.  \\ \hline
      ASRI rd, rs1, uimm4 & 00111011 & $\mathrm{rd} \leftarrow \mathrm{rs1} \ggg \mathrm{uimm4}$ &
      Do right arithmetic shift rs1 value with width uimm4, and store to rd.  \\ \hline
    \end{tabularx}
\end{center}

\subsection{Conditional Branches}
\vspace{-0.2in}
\begin{center}
\begin{tabular}{S@{}R@{}R@{}S@{}R@{}O}
\\
\instbitrange{23}{16} &
\instbitrange{15}{12} &
\instbitrange{11}{8} &
\instbitrange{7}{6} &
\instbitrange{5}{0} \\
\hline
\multicolumn{1}{|c|}{simm10[7:0]} &
\multicolumn{1}{c|}{rs1} &
\multicolumn{1}{c|}{rs2} &
\multicolumn{1}{c|}{simm10[9:8]} &
\multicolumn{1}{c|}{opcode} \\
\hline
8 & 4 & 4 & 2 & 6 \\
simm10[7:0]  & src1 & src2 & simm10[9:8] & BEQ \\
simm10[7:0]  & src1 & src2 & simm10[9:8] & BNE \\
simm10[7:0]  & src1 & src2 & simm10[9:8] & BLT \\
simm10[7:0]  & src1 & src2 & simm10[9:8] & BLTU \\
simm10[7:0]  & src1 & src2 & simm10[9:8] & BLE \\
simm10[7:0]  & src1 & src2 & simm10[9:8] & BLEU \\
\end{tabular}
\end{center}

\begin{center}
    \begin{tabularx}{\textwidth}{|c|c|X|} \hline
      Instruction & Opcode & \multicolumn{1}{c|}{Description} \\ \hline \hline
      BEQ rs1, rs2, simm10 & 001111 &
      If rs1 value is equal to rs2 value, jump to PC + simm10.  \\ \hline
      BNE rs1, rs2, simm10 & 101111 &
      If rs1 value is not equal to rs2 value, jump to PC + simm10.  \\ \hline
      BLT rs1, rs2, simm10 & 110111 &
      If rs1 signed value is less than rs2 one, jump to PC + simm10.  \\ \hline
      BLTU rs1, rs2, simm10 & 010111 &
      If rs1 unsigned value is less than rs2 one, jump to PC + simm10.  \\ \hline
      BLE rs1, rs2, simm10 & 111111 &
      If rs1 signed value is equal or less than rs2 one, jump to PC + simm10.  \\ \hline
      BLEU rs1, rs2, simm10 & 011111 &
      If rs1 unsigned value is equal or less than rs2 one, jump to PC + simm10.  \\ \hline
    \end{tabularx}
\end{center}

\subsection{Load and Store Instructions}
\vspace{-0.2in}
\begin{center}
\begin{tabular}{S@{}R@{}R@{}S@{}R@{}O}
\\
\instbitrange{23}{16} &
\instbitrange{15}{12} &
\instbitrange{11}{8} &
\instbitrange{7}{6} &
\instbitrange{5}{0} \\
\hline
\multicolumn{1}{|c|}{simm10[7:0]} &
\multicolumn{1}{c|}{rs} &
\multicolumn{1}{c|}{rd} &
\multicolumn{1}{c|}{simm10[9:8]} &
\multicolumn{1}{c|}{opcode} \\
\hline
8 & 4 & 4 & 2 & 6 \\
simm10[7:0]  & src & dest & simm10[9:8] & LW \\
simm10[7:0]  & src & dest & simm10[9:8] & LB \\
simm10[7:0]  & src & dest & simm10[9:8] & LBU \\
\end{tabular}
\end{center}

\begin{center}
    \begin{tabularx}{\textwidth}{|c|c|c|X|} \hline
      Instruction & Opcode & Formula & \multicolumn{1}{c|}{Description} \\ \hline \hline
      LW rd, simm10(rs)  & 010101 & $\mathrm{rd} \leftarrow [\mathrm{rs} + \mathrm{simm10}]$ &
      Load word value with address rs value + simm10, and store to rd. \\ \hline
      LB rd, simm10(rs)  & 100101 & $\mathrm{rd} \leftarrow [\mathrm{rs} + \mathrm{simm10}]$ &
      Load byte value with address rs value + simm10, and store sign extended one to rd.  \\ \hline
      LBU rd, simm10(rs) & 000101 & $\mathrm{rd} \leftarrow [\mathrm{rs} + \mathrm{simm10}]$ &
      Load byte value with address rs value + simm10, and store to rd without sign extension.  \\ \hline
    \end{tabularx}
\end{center}

\vspace{-0.2in}
\begin{center}
\begin{tabular}{S@{}R@{}R@{}S@{}R@{}O}
\\
\instbitrange{23}{16} &
\instbitrange{15}{12} &
\instbitrange{11}{8} &
\instbitrange{7}{6} &
\instbitrange{5}{0} \\
\hline
\multicolumn{1}{|c|}{simm10[7:0]} &
\multicolumn{1}{c|}{rd} &
\multicolumn{1}{c|}{rs} &
\multicolumn{1}{c|}{simm10[9:8]} &
\multicolumn{1}{c|}{opcode} \\
\hline
8 & 4 & 4 & 2 & 6 \\
simm10[7:0]  & dest & src & simm10[9:8] & SW \\
simm10[7:0]  & dest & src & simm10[9:8] & SB \\
\end{tabular}
\end{center}

\begin{center}
    \begin{tabularx}{\textwidth}{|c|c|c|X|} \hline
      Instruction & Opcode & Formula & \multicolumn{1}{c|}{Description} \\ \hline \hline
      SW rs, simm10(rd)  & 001101 & $[\mathrm{rd} + \mathrm{simm10}] \leftarrow \mathrm{rs}$ &
      Load value from rs, and store word value to address rs value + simm10.  \\ \hline
      SB rs, simm10(rd)  & 110101 & $[\mathrm{rd} + \mathrm{simm10}] \leftarrow \mathrm{rs}$ &
      Load value from rs, and store value[7:0] to address rs value + simm10.  \\ \hline
    \end{tabularx}
\end{center}

\section{16-bit Length Instructions}
\subsection{Integer Computational Instructions}
\subsubsection*{Integer Register-Register Instructions}
\vspace{-0.2in}
\begin{center}
\begin{tabular}{S@{}R@{}R@{}S@{}R@{}O}
\\
\instbitrange{15}{12} &
\instbitrange{11}{8} &
\instbitrange{7}{0} \\
\hline
\multicolumn{1}{|c|}{rs} &
\multicolumn{1}{c|}{rd} &
\multicolumn{1}{c|}{opcode} \\
\hline
4 & 4 & 8 \\
src  & dest & MOV \\
src  & dest & ADD2 \\
src  & dest & SUB2 \\
src  & dest & AND2 \\
src  & dest & XOR2 \\
src  & dest & OR2 \\
src  & dest & LSL2 \\
src  & dest & LSR2 \\
src  & dest & ASR2 \\
\end{tabular}
\end{center}

\begin{center}
    \begin{tabularx}{\textwidth}{|c|c|c|X|} \hline
      Instruction & Opcode & Formula & \multicolumn{1}{c|}{Description} \\ \hline \hline
      MOV rd, rs  & 11000000 & $\mathrm{rd} \leftarrow \mathrm{rs}$ &
      Load rs value, and store to rd. \\ \hline
      ADD2 rd, rs & 10000000 & $\mathrm{rd} \leftarrow \mathrm{rd} + \mathrm{rs}$ &
      Add rs1 value to rd value, and store to rd.  \\ \hline
      SUB2 rd, rs & 10001000 & $\mathrm{rd} \leftarrow \mathrm{rd} - \mathrm{rs}$ &
      Subtract rd value from rs value, and store to rd.  \\ \hline
      AND2 rd, rs & 10010000 & $\mathrm{rd} \leftarrow \mathrm{rd} \land \mathrm{rs}$ &
      Do bit and between rd value and rs value, and store to rd.  \\ \hline
      XOR2 rd, rs & 10011000 & $\mathrm{rd} \leftarrow \mathrm{rd} \oplus \mathrm{rs}$ &
      Do bit xor between rd value and rs value, and store to rd.  \\ \hline
      OR2  rd, rs & 10100000 & $\mathrm{rd} \leftarrow \mathrm{rd} \lor \mathrm{rs}$ &
      Do bit or between rd value and rs value, and store to rd.  \\ \hline
      LSL2 rd, rs & 10101000 & $\mathrm{rd} \leftarrow \mathrm{rd} \ll \mathrm{rs}$ &
      Do left logical shift rd value with width rs value, and store to rd.  \\ \hline
      LSR2 rd, rs & 10110000 & $\mathrm{rd} \leftarrow \mathrm{rd} \gg \mathrm{rs}$ &
      Do right logical shift rd value with width rs value, and store to rd.  \\ \hline
      ASR2 rd, rs & 10111000 & $\mathrm{rd} \leftarrow \mathrm{rd} \ggg \mathrm{rs}$ &
      Do right arithmetic shift rd value with width rs value, and store to rd.  \\ \hline
    \end{tabularx}
\end{center}

\subsubsection*{Integer Register-Immediate Instructions}

\vspace{-0.2in}
\begin{center}
\begin{tabular}{S@{}R@{}R@{}S@{}R@{}O}
\\
\instbitrange{15}{12} &
\instbitrange{11}{8} &
\instbitrange{7}{6} &
\instbitrange{5}{0} \\
\hline
\multicolumn{1}{|c|}{simm6[3:0]} &
\multicolumn{1}{c|}{rd} &
\multicolumn{1}{|c|}{simm6[5:4]} &
\multicolumn{1}{c|}{opcode} \\
\hline
4 & 4 & 2 & 6 \\
simm6[3:0]  & dest &  simm6[5:4] & ADDI2 \\
simm6[3:0]  & dest &  simm6[5:4] & ANDI2 \\
simm6[3:0]  & dest &  simm6[5:4] & LSI \\
simm6[3:0]  & dest &  simm6[5:4] & LUI \\
\end{tabular}
\end{center}

\begin{center}
    \begin{tabularx}{\textwidth}{|c|c|c|X|} \hline
      Instruction & Opcode & Formula & \multicolumn{1}{c|}{Description} \\ \hline \hline
      ADDI2 rd, simm6 & 000010 & $\mathrm{rd} \leftarrow \mathrm{rd} + \mathrm{simm6}$ &
      Add rd value to simm6, and store to rd.  \\ \hline
      ANDI2 rd, simm6 & 010010 & $\mathrm{rd} \leftarrow \mathrm{rd} \land \mathrm{simm6}$ &
      Do bit and between rd value and simm6, and store to rd.  \\ \hline
      LSI rd, simm6   & 110100 & $\mathrm{rd} \leftarrow \mathrm{simm6}$ &
      Load simm6, and store to rd. \\ \hline
      LUI rd, simm6   & 000100 & $\mathrm{rd} \leftarrow (\mathrm{simm6} \ll \mathrm{10})$ &
      Load simm6 with left logical shift with width 10, and store to rd.  \\ \hline
    \end{tabularx}
\end{center}

\vspace{-0.2in}
\begin{center}
\begin{tabular}{S@{}R@{}R@{}S@{}R@{}O}
\\
\instbitrange{15}{12} &
\instbitrange{11}{8} &
\instbitrange{7}{0} \\
\hline
\multicolumn{1}{|c|}{uimm4[3:0]} &
\multicolumn{1}{c|}{rd} &
\multicolumn{1}{c|}{opcode} \\
\hline
4 & 4 & 8 \\
uimm4[3:0]  & dest & LSLI2 \\
uimm4[3:0]  & dest & LSRI2 \\
uimm4[3:0]  & dest & ASRI2 \\
\end{tabular}
\end{center}

\begin{center}
    \begin{tabularx}{\textwidth}{|c|c|c|X|} \hline
      Instruction & Opcode & Formula & \multicolumn{1}{c|}{Description} \\ \hline \hline
      LSLI2 rd, uimm4 & 00101010 & $\mathrm{rd} \leftarrow \mathrm{rd} \ll \mathrm{uimm4}$ &
      Do left logical shift rd value with width uimm4, and store to rd.  \\ \hline
      LSRI2 rd, uimm4 & 00110010 & $\mathrm{rd} \leftarrow \mathrm{rd} \gg \mathrm{uimm4}$ &
      Do right logical shift rd value with width uimm4, and store to rd.  \\ \hline
      ASRI2 rd, uimm4 & 00111010 & $\mathrm{rd} \leftarrow \mathrm{rd} \ggg \mathrm{uimm4}$ &
      Do right arithmetic shift rd value with width uimm4, and store to rd.  \\ \hline
    \end{tabularx}
\end{center}

\subsection{Unconditional Jumps}
\vspace{-0.2in}
\begin{center}
\begin{tabular}{S@{}R@{}R@{}S@{}R@{}O}
\\
\instbitrange{15}{12} &
\instbitrange{11}{8} &
\instbitrange{7}{0} \\
\hline
\multicolumn{1}{|c|}{0000} &
\multicolumn{1}{c|}{rs} &
\multicolumn{1}{c|}{opcode} \\
\hline
4 & 4 & 8 \\
& src & JALR \\
& src & JR \\
\end{tabular}
\end{center}

\begin{center}
    \begin{tabularx}{\textwidth}{|c|c|c|X|} \hline
      Instruction & Opcode & Formula & \multicolumn{1}{c|}{Description} \\ \hline \hline
      JALR rs     & 00010110 & \begin{tabular}{@{}c@{}}$\mathrm{ra} \leftarrow \mathrm{PC} + 2$\\$\mathrm{PC} \leftarrow \mathrm{rs}$\end{tabular} &
      Jump to rs value, and store PC + 2 to return address register ra.  \\ \hline
      JR rs       & 00000110 & $\mathrm{PC} \leftarrow \mathrm{rs}$ &
      Jump to rs value. \\ \hline
    \end{tabularx}
\end{center}

\vspace{-0.2in}
\begin{center}
\begin{tabular}{S@{}R@{}R@{}S@{}R@{}O}
\\
\instbitrange{15}{5} &
\instbitrange{4}{0} \\
\hline
\multicolumn{1}{|c|}{simm11[10:0]} &
\multicolumn{1}{c|}{opcode} \\
\hline
11 & 5 \\
simm11[10:0]  & JS \\
simm11[10:0]  & JSAL \\
\end{tabular}
\end{center}

\begin{center}
    \begin{tabularx}{\textwidth}{|c|c|c|X|} \hline
      Instruction & Opcode & Formula & \multicolumn{1}{c|}{Description} \\ \hline \hline
      JS simm11   & 01110    & $\mathrm{PC} \leftarrow \mathrm{PC} + \mathrm{simm11}$ &
      Jump to PC + simm11. \\ \hline
      JSAL simm11 & 11110    & \begin{tabular}{@{}c@{}}$\mathrm{ra} \leftarrow \mathrm{PC} + 2$\\$\mathrm{PC} \leftarrow \mathrm{PC} + \mathrm{simm11}$\end{tabular} &
      Jump to PC + simm11, and store PC + 2 to return address register ra.  \\ \hline
    \end{tabularx}
\end{center}

\subsection{Load and Store Instructions}
\vspace{-0.2in}
\begin{center}
\begin{tabular}{S@{}R@{}R@{}S@{}R@{}O}
\\
\instbitrange{15}{12} &
\instbitrange{11}{8} &
\instbitrange{7}{6} &
\instbitrange{5}{0} \\
\hline
\multicolumn{1}{|c|}{uimm7[4:1]} &
\multicolumn{1}{|c|}{rd} &
\multicolumn{1}{|c|}{uimm7[6:5]} &
\multicolumn{1}{c|}{opcode} \\
\hline
11 & 5 \\
uimm7[4:1] & dest & uimm7[6:5] & LWSP \\
\end{tabular}
\end{center}

\begin{center}
    \begin{tabularx}{\textwidth}{|c|c|c|X|} \hline
      Instruction & Opcode & Formula & \multicolumn{1}{c|}{Description} \\ \hline \hline
      LWSP rd, uimm7(sp) & 010100 & $\mathrm{rd} \leftarrow [\mathrm{sp} + \mathrm{uimm7}]$ &
      Load word value with address stack pointer   register value + uimm7, and store to rd.  \\ \hline
    \end{tabularx}
\end{center}

\vspace{-0.2in}
\begin{center}
\begin{tabular}{S@{}R@{}R@{}S@{}R@{}O}
\\
\instbitrange{15}{12} &
\instbitrange{11}{8} &
\instbitrange{7}{6} &
\instbitrange{5}{0} \\
\hline
\multicolumn{1}{|c|}{uimm7[4:1]} &
\multicolumn{1}{|c|}{rs} &
\multicolumn{1}{|c|}{uimm7[6:5]} &
\multicolumn{1}{c|}{opcode} \\
\hline
11 & 5 \\
uimm7[4:1] & src & uimm7[6:5] & SWSP \\
\end{tabular}
\end{center}

\begin{center}
    \begin{tabularx}{\textwidth}{|c|c|c|X|} \hline
      Instruction & Opcode & Formula & \multicolumn{1}{c|}{Description} \\ \hline \hline
      SWSP rs, uimm7(sp) & 011100 & $[\mathrm{sp} + \mathrm{uimm7}] \leftarrow \mathrm{rs}$ &
      Store rs value to uimm8 with   address stack pointer register value + uimm7.  \\ \hline
    \end{tabularx}
\end{center}

\section{Instructions List}

\end{document}